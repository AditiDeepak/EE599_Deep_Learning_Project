%% LyX 2.3.5.2 created this file.  For more info, see http://www.lyx.org/.
%% Do not edit unless you really know what you are doing.
\documentclass[11pt]{article}
\usepackage[latin9]{inputenc}
\usepackage{amsmath}
\usepackage{amssymb}
\usepackage[unicode=true,
 bookmarks=false,
 breaklinks=false,pdfborder={0 0 1},backref=section,colorlinks=false]
 {hyperref}
\hypersetup{pdftitle={EE599ProjectSummary},
 pdfauthor={BrandonFranzke},
 pdftex,bookmarks,colorlinks,citecolor=green,filecolor=Orange,linkcolor=blue,urlcolor=BrickRed,pdftex}

\makeatletter
%%%%%%%%%%%%%%%%%%%%%%%%%%%%%% User specified LaTeX commands.
\usepackage{graphicx}




\usepackage[html,dvipsnames]{xcolor}



\setlength{\textwidth}{6.5in}
\setlength{\textheight}{9.0in}
\headheight=0.5in
\topmargin=-0.75in
\oddsidemargin= 0.0in
\evensidemargin=-0.25in


 




\markright{{\bf EE599 - \copyright B. Franzke - Fall 2020} }


\title{\bf EE599 Deep Learning -- Initial Project Proposal}
\author{\copyright  B. Franzke}

\makeatother

\begin{document}
\maketitle

\paragraph{Project Title:}

Realtime Background Replacement and Super Resolution for Video Conferencing
Applications

\paragraph{Project Team:}

Adityan Jothi, Aditi Hoskere Deepak, Srujana Subramanya

\paragraph{Project Summary:}

In this project we propose to generate photo-realistic background
replaced images in real-time along with super resolution for video
conference applications using GANs. We will collect data from Youtube-8M
dataset and use some Zoom call recordings amongst the group. The project
will involve experimenting on different GAN architectures and loss
functions for Super Resolution and LSTM for Background Replacement.
It would also entail a way to transmit the encoded latent vector between
two applications to reconstruct the image with improved quality in
real-time. A successful outcome would be to produce a 4-5 minute video
call that is photorealistic with improved image quality. 

\paragraph{Data Needs and Acquisition Plan:}

We would be using videos similar to teleconferencing calls, interviews,
Twitch game streams, and more from the Youtube-8M dataset as a baseline
and include several recorded videos of Zoom calls amongst the group.
For the dataset generated by us, we would build an auto annotation
tool using segmentation model to isolate subject efficiently in the
video clips for labeling. The dataset is available open-source mapped
to YouTube videos that can be downloaded using the youtube downloader
tool.

\paragraph{Primary References and Codebase:}

We propose to build on the approach used in
\begin{itemize}
\item Olof Mogren, ``\href{https://arxiv.org/pdf/1609.04802.pdf}{Photo-Realistic Single Image Super-Resolution Using a Generative Adversarial Network},''
Computer Vision and Pattern Recognition (CVPR).
\item Architecture: \href{https://arxiv.org/pdf/1505.04597.pdf}{U-Net paper},
\item GitHub codebases: \href{https://github.com/eti-p-doray/unet-gan-matting}{Background Removal},
\href{https://github.com/HasnainRaz/Fast-SRGAN}{Fast-SRGAN Code}
\end{itemize}

\paragraph{Architecture Investigation Plan:}

We plan to use U-Net with GANs for Background Replacement and SRGAN for Image Super Resolution. Then, we will explore various neural network architectures, hyperparameters to improve the accuracy of our models.    

\paragraph{Estimated Compute Needs:}

Based on the data set size in the above paper and the benchmarks in this original U-Net paper, we estimate that one training run for our initial U-Net GAN architecture will take 20 hours on a single nVidia V100 GPU, which is the GPU resource in the AWS p3.2xlarge instance. With spot pricing, which is roughly $1 per hour, we expect $20 per training run. For the SRGAN architecture we estimate that one training run will take 15 hours; so we estimation $15 per run. We expect to do a number of provisional runs to tune hyper-parameters. We estimate that this will cost approximate $40. In addition, we expect to do approximately 4 full runs which brings our total estimated computing cost to roughly $250. Pooling our resources, we expect to be about $100 short of this value. 

\paragraph{Team Roles:}

The following is the rough breakdown of roles and responsibilities
we plan for our team:

\begin{itemize}
\item Adityan: Super Resolution GAN 
\item Srujana: Background Replacement U-Net,Auto-annotation model(Data collection and cleaning)
\item Aditi:   Background Replacement GAN, Auto-annotation model(Data collection and cleaning)
\end{itemize}
All team members will work on the final presentation, slides, and
report.

\paragraph{Requested Mentor with Rationale:}

We request Professor Franzke to be our team mentor because he has
expertise in Computer Vision and GANs. Oliver is our second choice
because of his expertise in GANs. We have a good idea of what we want
to do and have a good starting point from the paper and codebase,
so we are flexible regarding our mentor assignment.
\end{document}
